\documentclass{article}
\usepackage[UTF8]{ctex}
\usepackage{amsmath}
\usepackage{hyperref}
\usepackage{titlesec}
\usepackage{amsthm}
\usepackage[total={7in,10in}]{geometry}
\usepackage{mathtools}
\usepackage{siunitx}
\usepackage{tikz}
\usepackage{caption}
\usepackage[RPvoltages]{circuitikz}

\usetikzlibrary{arrows}

\newcounter{para}
\newcommand\mypara{\par\refstepcounter{para}(\thepara)\space}
\titleformat{\section}[block]{\Large\bfseries\filcenter}{}{0em}{}
\renewcommand\thesection{}
\renewcommand\thesubsection{\protect\setcounter{equation}{0}\protect\setcounter{para}{0}第 \arabic{subsection} 题}

\title{hs\_phys\_probs 004}
\author{詹有丘}
\date{}

\begin{document}

\maketitle

\subsection{突破}

考虑质量为 $m$ 的非相对论性粒子在势场 $U=U\!\left(x\right)$ 中的一维运动.

\mypara
若 $U=-Ax^4$, 其中 $A>0$,
证明具有非零能量的粒子总会在有限时间内到达无穷远处.

\mypara
若 $U$ 有下界, 证明粒子不能在有限时间内到达无穷远处.

\subsection{惯量椭球}

给定一个刚体. 选取点 O.
任取以 O 为起点的射线, 在射线上取点 P,
使得 $\left|\mathrm{OP}\right|$ 等于刚体绕 OP 轴的转动惯量的 $\gamma$ 次幂.
$\gamma$ 被恰当地选取, 使得对于任意的 O, P 的轨迹是椭球面.
证明该椭球的体积取到最值当且仅当 O 是该刚体的质心.

\subsection{圆轨道常量}

粒子在有心力场中运动.
记粒子在 $r_{\mathrm{min}}$ 和 $r_{\mathrm{max}}$
之间往复运动一个周期的过程中位矢转过的角度为 $\Phi$.
显然它是粒子的能量和角动量的函数.
现调整其能量和角动量, 使得粒子的运动轨迹趋近于圆轨道,
记这一过程中 $\Phi$ 的极限为 $\Psi$.
若 $\Psi$ 与轨道的半径无关, 求该有心力场的形式.

\subsection{摆线气体}

粒子数为 $N$ 的一些单原子气体被约束在摆线
$$\begin{dcases}x=R\left(\varphi-\sin\varphi\right),\\
y=R\cos\varphi,\end{dcases}\quad\varphi\in\left[0,2\pi\right]$$
上. 这些气体在温度 $T$ 下达到热平衡状态.
每个微粒受到 $-y$ 方向的重力, 其大小为 $\frac{kT}{R}$,
其中 $k$ 为 Boltzmann 常量.
求这些气体的质心坐标.
可能有用的特殊函数:
\begin{equation*}
	\operatorname{erf}z\coloneqq\frac2{\sqrt\pi}\int_0^z\mathrm e^{-t^2}\mathrm dt.
\end{equation*}

\subsection{半反半透膜}

真空中有一块平整的厚度为 $d$ 的薄膜,
其相对介电常数为 $\varepsilon_\mathrm r$,
相对磁导率为 $\mu_\mathrm r$.
求其能量反射率 $R$,
表达为入射角 $\theta$ 和偏振方向 $\alpha$
(电场强度与法线平面所夹锐角, $0$ 对应 p 偏振态, $\frac\pi2$ 对应 s 偏振态) 的函数.

\subsection{电子偶素}

求电子偶素 (如同氢原子, 但由正电子代替原子核) 的结合能, 精确到 $\SI{0.1}{eV}$.

\subsection{电电圈}

有一个圆环 (甜甜圈), 截面的圆心到圆环的轴线的距离为 $R$, 截面的半径为 $r\ll R$,
圆环内均匀分布着总量为 $Q$ 的电荷.
圆环以角速度 $\omega$ 绕着轴线旋转.
求轴线附近的磁感应强度,
用柱坐标表示, 精确到 $\rho$ 的一阶和 $r$ 的二阶.

\subsection{Theremin}

1928 年, 发明家 Leon Theremin 发明了乐器 theremin.
这是一种不需要身体接触就可以演奏的电子乐器.
现介绍其基本原理.
它具有两个天线, 分别被称为音调天线和音量天线.
演奏者的右手与音调天线组成一个电容器, 演奏者的左手与音量天线组成一个电容器,
而演奏者的脚接地.
当演奏者的手移动时, 电容就会发生改变, 从而导致振荡电路的振荡频率发生改变.
振荡频率不断改变的电信号在经过电子电路的处理之后被输入压控放大器,
输出最终能被扬声器播放的音频信号.
现探究演奏者是如何控制音调的.

\mypara
在音调信号被送入压控放大器之前,
会先通过低通滤波器, 将人听不见的超声波滤去.
一个简单的低通滤波器由电阻和电容器串联组成,
输入电压施加在串联 RC 的两端, 输出电压即为电容器上的分压.
已知人能听见的声音的最高频率为 $f_{\mathrm{max}}$,
若要保证超声波的振幅通过率不超过振幅通过率最高的频段的振幅通过率的 $\eta$ 倍,
求 $R$ 和 $C$ 满足的条件.

\mypara
被送入低通滤波器的音调信号是由可变音调振荡器的输出信号
$U_1=U_{10}\cos\!\left(2\pi f_1t\right)$
和参考音调振荡器的输出信号
$U_2=U_{20}\cos\!\left(2\pi f_2t\right)$
经过混频器混合而成的,
它们的频率满足 $f_1\approx f_2\gg f_{\mathrm{max}}$,
$\left|f_1-f_2\right|<f_{\mathrm{max}}$.
一种简易的混频器只需要一个二极管
(伏安特性为 $I=I_0\left(\mathrm e^{U/U_0}-1\right)$,
其中 $U_0\gg U_{10},U_{20}$)
和一个电阻 $R\ll U_0/I_0$ 串联即可制成.
两个输入信号被串联后施加在二极管和电阻上,
输出信号即为电阻上的分压.
求混频器输出的信号中的低频部分 (高频部分会被低通滤波器滤除, 无需写出).

\mypara
可变音调振荡器和参考音调振荡器都是 LC 振荡电路,
其中可变音调振荡器的电容由音调天线和演奏者的右手作为两个电极,
而参考音调振荡器在 theremin 的内部且保持不变.
振荡器利用互感向混频器输出信号.
根据以上所给信息画出 theremin 的音调控制部分的电路图
(无需画出音量控制部分和扬声器),
标注演奏者, 音调天线, 可变音调振荡器, 参考音调振荡器, 混频器, 低通滤波器.

\mypara
将音调天线和演奏者的右手看作轴线互相平行的两个长度为 $l$, 半径分别为
$R_1$ 和 $R_2$ 的圆柱形导体.
在演奏过程中, 这两个圆柱的两端的连线始终垂直于它们的轴线,
而它们的轴线间距在 $d_{\mathrm{min}}$ 和 $d_{\mathrm{max}}$ 之间改变,
从而改变 $f_1$ (不是已知量).
已知 $l\gg R_1,R_2,d_{\mathrm{min}},d_{\mathrm{max}}$.
若低通滤波器最终输出的信号的频率刚好能覆盖人能听见的频段
($f_{\mathrm{min}}$ 到 $f_{\mathrm{max}}$ 之间),
求可变音调振荡器中的电感 $L_1$ 和参考音调振荡器的频率 $f_2$.

\newpage
\section{参考答案}

\subsection{突破}

\mypara
\begin{proof}
显然原问题等价为证明
\begin{equation}
	I:=\int_1^{+\infty}\frac{\mathrm du}{\sqrt{u^4-1}}
\end{equation}
收敛 (给出这一等价命题可得一半分).
令
\begin{equation}
	I_1:=\int_1^2\frac{\mathrm du}{\sqrt{u^4-1}},
	\qquad I_2:=\int_2^{+\infty}\frac{\mathrm du}{\sqrt{u^4-1}},
\end{equation}
则有 $I=I_1+I_2$.
现在分别处理 $I_1$ 和 $I_2$.

首先处理 $I_1$.
注意到, 当 $u>1$ 时, 有
\begin{equation}
	u^4-1=\left(u^2+1\right)\left(u+1\right)\left(u-1\right)
	>4\left(u-1\right),
\end{equation}
因此
\begin{equation}
	I_1<\int_1^2\frac{\mathrm du}{\sqrt{4\left(u-1\right)}}=1.
\end{equation}
由比较审敛法知 $I_1$ 收敛.

再处理 $I_2$.
注意到, 当 $u>2$ 时, 有
\begin{equation}
	u^4-2>14>0\Rightarrow u^4-1>\frac12u^4,
\end{equation}
因此
\begin{equation}
	I_2<\int_2^{+\infty}\frac{\mathrm du}{\sqrt{\frac12u^4}}=\frac1{\sqrt2}.
\end{equation}
由比较审敛法知 $I_2$ 收敛.

因此 $I$ 收敛, 从而任意具有非零能量的粒子都会在有限时间内到达无穷远.
\end{proof}

\textit{推广}: 对于 $A,n>0$,
势场 $U=-A\left|x\right|^n$ 中具有非零能量的粒子能在有限时间内到达无穷远处,
当且仅当 $n>2$.

\begin{proof}
问题等价于证明
\begin{equation}
	I:=\int_1^{+\infty}\frac{\mathrm du}{\sqrt{u^n-1}}
\end{equation}
收敛, 当且仅当 $n>2$.
置换元 $w:=u^{-n}$, 则有
\begin{equation}
	I=\frac1n\int_0^1w^{-\frac12-\frac1n}\left(1-w\right)^{-\frac12}\mathrm dw
	=\frac1n\mathrm B\!\left(\frac12-\frac1n,\frac12\right),
\end{equation}
其中 $\mathrm B$ 是 beta 函数.
注意到在宗量的实部趋于 $0^+$ 时 beta 函数趋于正无穷,
可以得到积分收敛的条件
\begin{equation}
	\frac12-\frac1n>0.
\end{equation}
\end{proof}

\mypara
\begin{proof}
设 $U$ 有下界 $U_0$, 则具有能量 $E$ 的粒子在 $\tau$ 时间内的位移大小
\begin{equation}
\begin{split}
	\left|\int_0^\tau\dot x\,\mathrm dt\right|
	\le&\int_0^\tau\left|\dot x\right|\mathrm dt\\
	=&\int_0^\tau\sqrt{\frac2m\left(E-U\right)}\,\mathrm dt\\
	<&\int_0^\tau\sqrt{\frac2m\left(E-U_0\right)}\,\mathrm dt\\
	=&\,\tau\sqrt{\frac2m\left(E-U_0\right)},
\end{split}
\end{equation}
这在 $\tau$ 有限时不可能为无穷大量.
\end{proof}

\subsection{惯量椭球}

\begin{proof}
建立直角坐标系 $\mathrm O$, 使得三条轴对应于刚体的三个惯量主轴.

设 OP 方向的单位矢量为 $\mathbf n$.
设刚体以角速度 $\omega\mathbf n$ 转动,
且 $\omega$ 恰好使得其具有 $\frac12$ 单位能量的动能,
即
\begin{equation}
	\label{eq:用分量之和表示动能}
	E=\frac12\sum_jI_j\left(\omega n_j\right)^2=\frac12
\end{equation}
另一方面, 刚体具有的动能还可以表示为
\begin{equation}
	\label{eq:另一个动能}
	E=\frac12I\omega^2,
\end{equation}
其中 $I$ 是绕 OP 轴的转动惯量.
用式 \ref{eq:另一个动能} 表示出 $\omega$ 并代入式 \ref{eq:用分量之和表示动能},
可得
\begin{equation}
	\label{eq:力学角度的椭球}
	\sum_jI_j\frac{n_j^2}I=1.
\end{equation}

另一方面, 由题目给出的椭球的构造方法, 可以得出椭球面的方程
\begin{equation}
	\label{eq:数学角度的椭球}
	\sum_j\frac{\left(I^\gamma n_j\right)^2}{a_j^2}=1,
\end{equation}
其中 $a_j$ 是椭球的第 $j$ 半轴长.
比较式 \ref{eq:力学角度的椭球} 与式 \ref{eq:数学角度的椭球},
它们应当描述同一个椭球.
由此可得
\begin{equation}
	\gamma=-\frac12,\qquad a_j=I_j^{-\frac12}.
\end{equation}
由此, 可以注意到椭球的体积正比于三个主转动惯量的乘积的 $\gamma$ 次幂.
注意到三个主转动惯量的乘积就是惯量矩阵的行列式 (因为矩阵的行列式等于其特征值之积),
因此椭球的体积取最值相当于惯量矩阵的行列式取最值.
另外, 为了保证这是个椭球, 我们可以得知对于任意的 O, 刚体的三个主转动惯量都不为零.

令 $\mathrm O'$ 为刚体的质心.
重新建立直角坐标系, 使得三条轴对应于刚体绕 $\mathrm O'$ 转动的三个惯量主轴.
从而, 刚体绕质心的惯量矩阵可以写为 
\begin{equation}
	I'=\left[\begin{matrix}a\\&b\\&&c\end{matrix}\right].
\end{equation}
设 $\mathrm O$ 在以 $\mathrm O'$ 为原点的坐标系中的坐标为 $\left(x_1,x_2,x_3\right)$.
利用平行轴定理, 可得
\begin{equation}
	I_{j,k}=I'_{j,k}+\Delta I_{j,k},
\end{equation}
其中
\begin{equation}
	\Delta I_{j,k}:=\delta_{j,k}\sum_lx_l^2-x_jx_k
\end{equation}
(不失一般性地假设了刚体具有单位质量).

现在我们要令 $\det_{j,k}I_{j,k}$ 取极值,
只需对每个 $l$, 让它对 $x_l$ 的偏导数等于零即可.

首先我们有 Jacobi 公式 (利用 Laplace 展开可以容易地证明这一公式)
\begin{equation}
	\label{eq:行列式对矩阵元的全微分}
	\mathrm d\det_{j,k}I_{j,k}=\sum_{j,k}I^*_{k,j}\,\mathrm dI_{j,k},
\end{equation}
其中 $I^*$ 是 $I$ 的伴随矩阵.
同时, 将 $I_{j,k}$ 对 $x_l$ 求导可得
\begin{equation}
	\frac{\partial I_{j,k}}{\partial x_l}=\frac{\partial}{\partial x_l}\Delta I_{j,k}=
	2\delta_{j,k}x_l-\delta_{j,l}x_j-\delta_{k,l}x_k.
\end{equation}
将其代入式 \ref{eq:行列式对矩阵元的全微分}, 可得
\begin{equation}
\begin{split}
	\mathrm d\det_{j,k}I_{j,k}&=\sum_{j,k}I^*_{k,j}
	\sum_l\left(2\delta_{j,k}x_l-\delta_{j,l}x_j-\delta_{k,l}x_k\right)\mathrm dx_l\\
	&=\sum_l\mathrm dx_l\sum_{j,k}I^*_{k,j}\left(2\delta_{j,k}x_l-\delta_{j,l}x_j-\delta_{k,l}x_k\right)\\
	&=\sum_l\mathrm dx_l
	\left(2x_l\sum_{j,k}\delta_{j,k}I^*_{k,j}
	-\sum_{j,k}\delta_{j,l}I^*_{k,j}x_j
	-\sum_{j,k}\delta_{k,l}I^*_{k,j}x_k\right)\\
	&=\sum_l\mathrm dx_l\left(2x_l\sum_jI^*_{j,j}-\sum_kI^*_{k,l}x_l-\sum_jI^*_{l,j}x_l\right)\\
	&=\sum_lx_l\,\mathrm dx_l\left(2\sum_jI^*_{j,j}-\sum_kI^*_{k,l}-\sum_jI^*_{l,j}\right)\\
	&=2\sum_lx_l\,\mathrm dx_l\sum_j\left(I^*_{j,j}-I^*_{j,l}\right)
\end{split}
\end{equation}
(最后一个等号是因为 $I^*$ 作为对称矩阵 $I$ 的伴随矩阵, 必然是对称矩阵).
由该全微分式可以得到
\begin{equation}
	\frac{\partial}{\partial x_l}\det_{j,k}I_{j,k}=2x_l\sum_j\left(I^*_{j,j}-I^*_{j,l}\right).
\end{equation}

在 $\det_{j,k}I_{j,k}$ 取极值时, 其对各个 $x_l$ 的偏导数为 $0$.
由此可得, 对于每个 $l$, 有
\begin{equation}
	x_l=0\text{ 或 }\sum_j\left(I^*_{j,j}-I^*_{j,l}\right)=0
\end{equation}
成立.

情形 1: 对每个 $l$, 都有 $x_l=0$.

该情形显然可以成立.
若只有该情形成立, 原命题天然成立.

情形 2: 有且只有两个 $l$, 使得 $x_l=0$.

我们将证明该情形不可能成立.
将 $x_1,x_2,x_3$ 记作 $x,y,z$ 以简化算式.
不失一般性地, 假设 $x\ne0$ 且 $y=z=0$.
在这种情况下将 $I$ 显式地以矩阵写出, 得
\begin{equation}
	I=\left[\begin{matrix}a\\&b+x^2\\&&c+x^2\end{matrix}\right].
\end{equation}
由于 $x\ne0$, 目标行列式对 $x$ 的偏导数等于零要求
\begin{equation}
	\sum_j\left(I^*_{j,j}-I^*_{j,1}\right)=0,
\end{equation}
即
\begin{equation}
	I^*_{2,2}+I^*_{3,3}=I^*_{2,1}+I^*_{3,1}.
\end{equation}
由此可得
\begin{equation}
	a\left(c+x^2\right)+a\left(b+x^2\right)=0+0.
\end{equation}
这显然无解.

情形 3: 有且只有一个 $l$, 使得 $x_l=0$.

我们将证明该情形不可能成立.
不失一般性地, 假设 $x,z\ne0$ 且 $y=0$.
将 $I$ 显式地以矩阵写出, 得
\begin{equation}
	I=\left[\begin{matrix}a+z^2&&-xz\\&b+x^2+z^2\\-xz&&c+x^2\end{matrix}\right].
\end{equation}
由于 $x,z\ne0$, 目标行列式对 $x$ 和 $z$ 的偏导数等于零要求
\begin{equation}
	\begin{dcases}
		\sum_j\left(I^*_{j,j}-I^*_{j,1}\right)=0,\\
		\sum_j\left(I^*_{j,j}-I^*_{j,3}\right)=0,
	\end{dcases}
\end{equation}
即
\begin{equation}
	\label{eq:情形2方程组}
	\begin{dcases}
		I^*_{2,2}+I^*_{3,3}=I^*_{2,1}+I^*_{3,1},\\
		I^*_{1,1}+I^*_{2,2}=I^*_{1,3}+I^*_{2,3}.
	\end{dcases}
\end{equation}
将这两式相减, 注意到 $I^*_{2,1}=I^*_{2,3}=0$, 并且 $I^*_{1,3}=I^*_{3,1}$,
可得
\begin{equation}
	I^*_{1,1}=I^*_{3,3}.
\end{equation}
这意味着
\begin{equation}
	\label{eq:z用x表示}
	z^2=c-a+x^2.
\end{equation}

取式 \ref{eq:情形2方程组} 的第一个式子, 它可以显式地写为
\begin{equation}
	xz\left(b+x^2+z^2\right)=\left(a+z^2\right)\left(c+x^2\right)
	-x^2z^2+\left(a+z^2\right)\left(b+x^2+z^2\right).
\end{equation}
两边平方, 并将式 \ref{eq:z用x表示} 代入, 然后因式分解可得
\begin{equation}
	\left(\left(a+c\right)x^2+c^2\right)
	\left(8x^4+\left(-5a+6b+11c\right)x^2+\left(-a+b+2c\right)^2\right)=0.
\end{equation}
这两个因子中, 第一个因子不可能为零.
因此我们需要研究第二个因子能否为零.
第二个因子是关于 $x^2$ 的二次式, 它是否有零点取决于该二次式是否有非负根.
注意到该二次式的二次项系数和零次项系数都是非负的,
因此它有正根当且仅当一次项系数非正且判别式非负.
由此可以列出不等式组
\begin{equation}
	\begin{dcases}
		-5a+6b+11c\le0,\\
		\left(-5a+6b+11c\right)^2-32\left(-a+b+2c\right)^2\ge0.
	\end{dcases}
\end{equation}
该不等式组显然可以约化为一次不等式组
\begin{equation}
	-5a+6b+11c\le4\sqrt2\left(-a+b+2c\right)\le5a-6b-11c.
\end{equation}
该不等式组对于正的 $a,b,c$ 无解.

情形 4: 对每个 $l$, 都有 $x_l\ne0$.

我们将证明该情形不可能成立.
在这种情形下, 目标行列式的偏导数为零要求对于每个 $l$ 有
\begin{equation}
	\sum_jI^*_{j,j}=\sum_jI^*_{j,l}.
\end{equation}
可以将其以矩阵形式紧凑地写为
\begin{equation}
	\operatorname{tr}I^*\left[\begin{matrix}1\\1\\1\end{matrix}\right]=
	I^*\left[\begin{matrix}1\\1\\1\end{matrix}\right],
\end{equation}
这意味着 $\operatorname{tr}I^*$ 是 $I^*$ 的特征值.
由伴随矩阵的性质可知, $I^*$ 的特征值有 $ab,bc,ca$.
因此 $\operatorname{tr}I^*$ 等于 $ab,bc,ca$ 的其中之一.
这意味着
\begin{equation}
\begin{split}
	&\left(b+x^2+z^2\right)\left(c+x^2+y^2\right)-y^2z^2\\
	+&\left(a+y^2+z^2\right)\left(b+x^2+z^2\right)-x^2y^2\\
	+&\left(a+y^2+z^2\right)\left(c+x^2+y^2\right)-x^2z^2\\
	=&\,\text{$ab$ 或 $bc$ 或 $ca$}.
\end{split}
\end{equation}
这显然是无解的, 因为等式左边恒大于 $ab,bc,ca$ 中的任何一个.

综合以上四种情形, 只有 $x_l=0$ 才能使得目标行列式对所有 $x_l$ 的偏导数为零.
这意味着若 O 不与质心重合, 则惯量椭球的体积不可能取到局部极值.

最后证明 $x_l=0$ 确实是目标行列式的极值点.
为此我们需要考察目标行列式对 $x_l$ 的二阶导数.
经过繁琐的计算后可得
\begin{equation}
	\left.\frac{\partial^2}{\partial x_l\partial x_m}\det_{j,k}I_{j,k}\right|_0
	=2\left[\begin{matrix}a\left(b+c\right)\\&b\left(c+a\right)\\&&c\left(a+b\right)\end{matrix}\right].
\end{equation}
这显然是正定矩阵. 因此 $x_l=0$ 确实是目标行列式的极值.
\end{proof}

\subsection{圆轨道常量}

容易获得
\begin{equation}
	\Phi=\int_{r_\mathrm{min}}^{r_\mathrm{max}}\frac{\frac{M}{mr^2}\,\mathrm dr}{\sqrt{\frac2m\left(E-U-\frac{M^2}{2mr^2}\right)}}.
\end{equation}
代换 $x\coloneqq\frac{M}{mr}$, 注意到 $x_{\mathrm{min}}=\frac{M}{mr_{\mathrm{max}}}$ 以及 $x_{\mathrm{max}}=\frac{M}{mr_{\mathrm{min}}}$ 可得
\begin{equation}
	\Phi=\int_{x_{\mathrm{min}}}^{x_{\mathrm{max}}}\frac{\mathrm dx}{\frac2m\left(E-U-\frac{mx^2}2\right)}.
\end{equation}
于是形式上可以认为 $\Phi$ 是质量为 $m$ 的质点在势场
\begin{equation}
	W\!\left(x\right)\coloneqq U\!\left(\frac M{mx}\right)+\frac{mx^2}2
\end{equation}
中运动的半周期.

在圆轨道极限下, 该质点在势场 $W$ 的最低点微振动. 可以得到
\begin{equation}
	0=\lim_{\text{圆轨道}}W'\!\left(x\right)=\lim_{\text{圆轨道}}\left(-\frac M{mx^2}U'\!\left(\frac M{mx}\right)+mx\right)
	=\lim_{\text{圆轨道}}\left(-\frac{mr^2}MU'+\frac Mr\right),
\end{equation}
于是
\begin{equation}
	\label{eq:圆轨道下的M}
	\lim_{\text{圆轨道}}M=\sqrt{mr^3U'}.
\end{equation}
该微振动的半周期
\begin{equation}
	\label{eq:圆轨道下的Psi含M}
	\begin{split}
	\Psi\coloneqq\lim_{\text{圆轨道}}\Phi=\lim_{\text{圆轨道}}\pi\sqrt{\frac m{W''\!\left(x\right)}}&=
	\lim_{\text{圆轨道}}\pi\sqrt{\frac{m}{\frac{M^2}{m^2x^4}U''\!\left(\frac M{mx}\right)+\frac{2M}{mx^3}U'\!\left(\frac M{mx}\right)+m}}\\
	&=\lim_{\text{圆轨道}}\frac{\pi M}{\sqrt{mr^4U''+2mr^3U'^2+M^2}}.
	\end{split}
\end{equation}
将式 \ref{eq:圆轨道下的M} 代入式 \ref{eq:圆轨道下的Psi含M} 得
\begin{equation}
	\label{eq:关于U的微分方程}
	\Psi=\pi\sqrt{\frac{U'}{3U'+rU''}}.
\end{equation}

对于 $\Psi$ 与 $r$ 无关的情形, 将 $\Psi$ 视作常量, 将式 \ref{eq:关于U的微分方程} 视作关于 $U$ 的微分方程. 解得
\begin{equation}
	U=\begin{cases}
	B\ln\frac{r}{r_0},&\Psi=\frac\pi{\sqrt2},\\
	Ar^{\frac{\pi^2}{\Psi^2}-2}+C,&\Psi\ne\frac\pi{\sqrt2},
	\end{cases}
\end{equation}
其中 $A,B,C,r_0$ 为积分常量. 因此有心力场的形式为幂函数或对数函数.

\subsection{摆线气体}

由 Boltzmann 分布可知, 热平衡态下粒子数线密度 $n$ 正比于 $\mathrm e^{-\frac V{kT}}$,
其中势能
\begin{equation}
	V=mgy=\frac{kT}{R}y=kT\cos\varphi.
\end{equation}
于是
\begin{equation}
	\label{eq:数密度正比}
	n=A\mathrm e^{-\cos\varphi}.
\end{equation}
易知摆线上的线微元 $\mathrm ds=2R\sin\frac\varphi2\,\mathrm d\varphi$,
于是总粒子数
\begin{equation}
	N=2AR\int_0^{2\pi}\mathrm e^{-\cos\varphi}\sin\frac\varphi2\,\mathrm d\varphi.
\end{equation}
注意到
\begin{equation}
	\int_0^{2\pi}\mathrm e^{-\cos\varphi}\sin\frac\varphi2\,\mathrm d\varphi
	=-\sqrt2\int_0^{2\pi}\mathrm e^{1-2\cos^2\frac\varphi2}\,\mathrm d\!\left(\sqrt2\cos\frac\varphi2\right)
	=\mathrm e\sqrt2\int_{-\sqrt2}^{\sqrt2}\mathrm e^{-u^2}\,\mathrm du.
\end{equation}
利用题目所给 $\operatorname{erf}$ 的定义, 得
\begin{equation}
	N=2\mathrm e\sqrt{2\pi}\operatorname{erf}\sqrt2AR,
\end{equation}
于是
\begin{equation}
	A=\frac1{2\mathrm e\sqrt{2\pi}\operatorname{erf}\sqrt2}\frac NR.
\end{equation}
代入式 \ref{eq:数密度正比} 可知数密度, 于是知密度
\begin{equation}
	\lambda=\frac{nkT}{gR}=\frac1{2\mathrm e\sqrt{2\pi}\operatorname{erf}\sqrt2}\frac{NkT}{gR^2}\mathrm e^{-\cos\varphi}.
\end{equation}

于是易得质心纵坐标
\begin{equation}
	y_\mathrm c=\frac{gR}{NkT}\int_{\varphi=0}^{2\pi}y\lambda\,\mathrm ds=\frac R{\mathrm e\sqrt{2\pi}\operatorname{erf}\sqrt2}\int_0^{2\pi}\mathrm e^{-\cos\varphi}\cos\varphi\sin\frac\varphi2\,\mathrm d\varphi.
\end{equation}
注意到
\begin{equation}
	\int_0^{2\pi}\mathrm e^{-\cos\varphi}\cos\varphi\sin\frac\varphi2\,\mathrm d\varphi
	=-\sqrt2\int_0^{2\pi}\mathrm e^{1-2\cos^2\frac\varphi2}\left(2\cos^2\frac\varphi2-1\right)\mathrm d\!\left(\sqrt2\cos\frac\varphi2\right)=-\frac2{\mathrm e}-\mathrm e\sqrt{\frac\pi2}\operatorname{erf}\sqrt2.
\end{equation}
于是质心坐标为
\begin{equation}
	x_\mathrm c=\pi R,\qquad
	y_\mathrm c=\left(-\frac{\mathrm e^{-2}}{\operatorname{erf}\sqrt2}\sqrt\frac2\pi-\frac12\right)R.
\end{equation}

\textit{另}: 若不使用 Boltzmann 分布, 可对线微元应用静力平衡.

取线元 $\mathrm ds$.
易知其斜率为 $-\cot\frac\varphi2$.
线元两端的受到压力之差 (沿线元方向的压力的合力) 的大小为 $kT\,\mathrm dn$,
其应当等于该线元的重力在该方向上的分力 $\frac{kT}{R}n\,\mathrm ds\cdot\cos\frac\varphi2$.
于是得到微分方程
\begin{equation}
	kT\,\mathrm dn=n\frac{kT}R\cos\frac\varphi2\cdot2R\sin\frac\varphi2\,\mathrm d\varphi.
\end{equation}
由此可得
\begin{equation}
	n=A\mathrm e^{-\cos\varphi}.
\end{equation}
其与式 \ref{eq:数密度正比} 相同.

\subsection{半反半透膜}

介质的折射率为
\begin{equation}
	n\coloneqq\sqrt{\mu_\mathrm r\varepsilon_\mathrm r}.
\end{equation}
设入射表面为 $xy$ 平面.

设入射光线为
\begin{equation}
	\mathbf E=\mathbf E_0\mathrm e^{\mathrm i\left(\mathbf k\cdot\mathbf x-\omega t\right)},
	\quad\mathbf B=\frac1c\frac{\mathbf k\times\mathbf E}{k},
\end{equation}
折射光线为
\begin{equation}
	\mathbf E'=\mathbf E_0'\mathrm e^{\mathrm i\left(\mathbf k'\cdot\mathbf x-\omega t\right)},
	\quad\mathbf B=\frac nc\frac{\mathbf k'\times\mathbf E'}{k'},
\end{equation}
反射光线为
\begin{equation}
	\mathbf E''=\mathbf E_0''\mathrm e^{\mathrm i\left(\mathbf k''\cdot\mathbf x-\omega t\right)},
	\quad\mathbf B=\frac1c\frac{\mathbf k''\times\mathbf E'}{k''}.
\end{equation}
显然各光线的波数大小满足关系
\begin{equation}
	\label{eq:波数关系}
	k=k''=\frac\omega c,\qquad k'=\frac{\omega n}c.
\end{equation}

记反射角为 $r$, 由 Snell 定律 (或者由 $z=0$ 处相位因子相等) 可得
\begin{equation}
	\frac{\sin\theta}{\sin r}=n.
\end{equation}

由于 $z=0$ 边界处 $\mathbf E$ 与 $\mathbf H$ 的切向分量连续, 得到
\begin{equation}
	\label{eq:边界3}
	\left(\mathbf E_0+\mathbf E_0''-\mathbf E_0'\right)\times\mathbf n=0,
\end{equation}
\begin{equation}
	\label{eq:边界4}
	\left(\mu_\mathrm r\left(\mathbf k\times\mathbf E_0+\mathbf k''\times\mathbf E_0''\right)-\mathbf k'\times\mathbf E_0'\right)\times\mathbf n=0.
\end{equation}

将 $\mathbf E_0$ 分解为 s 偏振方向 $\mathbf E_{0\text s}$ 和 p 偏振方向 $\mathbf E_{0\text p}$.
对于 s 偏振, 式 \ref{eq:边界3} 给出
\begin{equation}
	E_{0\text s}+E_{0\text s}''-E_{0\text s}'=0.
\end{equation}
式 \ref{eq:边界4} 给出 (利用式 \ref{eq:波数关系})
\begin{equation}
	\sqrt{\mu_\mathrm r}\left(E_{0\text s}-E_{0\text s}''\right)\cos\theta-\sqrt{\varepsilon_\mathrm r}E_{0\text s}'\cos r=0.
\end{equation}
由此解得 s 偏振的折射率与反射率
\begin{equation}
	t_{1\text s}\coloneqq\frac{E_{0\text s}'}{E_{0\text s}}=\frac{2\mu_\mathrm r\cos\theta}{\mu_\mathrm r\cos\theta+\sqrt{n^2-\sin^2\theta}},
\end{equation}
\begin{equation}
	r_{1\text s}\coloneqq\frac{E_{0\text s}''}{E_{0\text s}}=\frac{\mu_\mathrm r\cos\theta-\sqrt{n^2-\sin^2\theta}}{\mu_\mathrm r\cos\theta+\sqrt{n^2-\sin^2\theta}}.
\end{equation}

对于 p 偏振, 式 \ref{eq:边界3} 给出
\begin{equation}
	\left(E_{0\text p}-E_{0\text p}''\right)\cos\theta-E_{0\text p}'\cos r=0.
\end{equation}
式 \ref{eq:边界4} 给出
\begin{equation}
	\sqrt{\mu_\mathrm r}\left(E_{0\text p}+E_{0\text p}''\right)-\sqrt{\varepsilon_\mathrm r}E_{0\text p}'=0.
\end{equation}
由此解得 p 偏振的折射率与反射率
\begin{equation}
	t_{1\text p}\coloneqq\frac{E_{0\text p}'}{E_{0\text p}}=\frac{2n\cos\theta}{n^2\cos\theta+\mu_\mathrm r\sqrt{n^2-\sin^2\theta}},
\end{equation}
\begin{equation}
	r_{1\text p}\coloneqq\frac{E_{0\text p}''}{E_{0\text p}}=\frac{n^2\cos\theta-\mu_\mathrm r\sqrt{n^2-\sin^2\theta}}{n^2\cos\theta+\mu_\mathrm r\sqrt{n^2-\sin^2\theta}}.
\end{equation}

类似地, 若从介质内部入射介质表面, 可得 s 偏振和 p 偏振的折射率与反射率
\begin{equation}
	t_{2\text s}=\frac{2\sqrt{n^2-\sin^2\theta}}{\sqrt{n^2-\sin^2\theta}+\mu_\mathrm r\cos\theta},
\end{equation}
\begin{equation}
	r_{2\text s}=\frac{\sqrt{n^2-\sin^2\theta}-\mu_{\mathrm r}\cos\theta}{\sqrt{n^2-\sin^2\theta}+\mu_{\mathrm r}\cos\theta},
\end{equation}
\begin{equation}
	t_{2\text p}=\frac{2\sqrt{n^2-\sin^2\theta}}{\mu_\mathrm r\sqrt{n^2-\sin^2\theta}+n\cos\theta},
\end{equation}
\begin{equation}
	r_{2\text p}=\frac{\mu_\mathrm r\sqrt{n^2-\sin^2\theta}-n\cos\theta}{\mu_\mathrm r\sqrt{n^2-\sin^2\theta}+n\cos\theta}.
\end{equation}

设在膜底面反射过 $l+1$ 次的反射光线的振幅为 $A_l$, 简单分析可知
\begin{equation}
	A_l=t_1t_2r_2^{2l+1}\mathrm e^{\mathrm in\frac{2d}{\cos r}l-\mathrm i\cdot2d\tan r\sin\theta\cdot l}E_0.
\end{equation}
干涉后反射光线的总振幅
\begin{equation}
	A=r_1\mathrm e^{\mathrm i\cdot2d\tan r\sin\theta}E_0+\sum_lA_l
	=\left(r_1\mathrm e^{2\mathrm id\tan r\sin\theta}+\frac{t_1t_2r_2}{1-r_2^2\mathrm e^{\mathrm in\frac{2d}{\cos r}-2\mathrm id\tan r\sin\theta}}\right)E_0.
\end{equation}
考虑到两种偏振, 于是有总能量反射率
\begin{equation}
	R=\frac{\left|A_\text s\right|^2\sin^2\alpha+\left|A_\text p\right|^2\cos^2\alpha}{E_0^2}.
\end{equation}
代入得到答案 (不必展开)
\begin{equation}
	R=
	\left|r_{1\text s}\mathrm e^{2\mathrm id\frac{\sin^2\theta}{\sqrt{n^2-\sin^2\theta}}}+\frac{t_{1\text s}t_{2\text s}r_{2\text s}}{1-r_{2\text s}^2\mathrm e^{2\mathrm id\frac{n-\sin^2\theta}{\sqrt{n^2-\sin^2\theta}}}}\right|^2\sin^2\alpha+
	\left|r_{1\text p}\mathrm e^{2\mathrm id\frac{\sin^2\theta}{\sqrt{n^2-\sin^2\theta}}}+\frac{t_{1\text p}t_{2\text p}r_{2\text p}}{1-r_{2\text p}^2\mathrm e^{2\mathrm id\frac{n-\sin^2\theta}{\sqrt{n^2-\sin^2\theta}}}}\right|^2\cos^2\alpha.
\end{equation}

\subsection{电子偶素}

一般地, 对于二体问题, 令 $\mathbf r\coloneqq\mathbf r_1-\mathbf r_2$, 以及 $\mathbf r_\mathrm c$ 为质心, 则有
\begin{equation}
	\mathbf r_1=\mathbf r_\mathrm c+\frac\mu{m_1}\mathbf r,\quad
	\mathbf r_2=\mathbf r_{\mathrm c}-\frac{\mu}{m_2}\mathbf r,
\end{equation}
其中 $\mu\coloneqq\frac{m_1m_2}{m_1+m_2}$ 为约化质量.
因此容易证明
\begin{equation}
	\label{eq:梯度}
	\nabla_1=\frac{\mu}{m_2}\nabla_\mathrm c+\nabla_r,\quad
	\nabla_2=\frac{\mu}{m_1}\nabla_{\mathrm c}-\nabla_r.
\end{equation}

设势场的形式为 $V\!\left(\mathbf r\right)$, 则不含时 Schrödinger 方程给出
\begin{equation}
	\label{eq:不含时SE}
	-\frac{\hbar^2}{2m_1}\nabla_1^2\psi-\frac{\hbar^2}{2m_2}\nabla_2^2\psi+V\!\left(\mathbf r\right)\psi=E\psi.
\end{equation}
将式 \ref{eq:梯度} 代入式 \ref{eq:不含时SE} 可得
\begin{equation}
	-\frac{\hbar^2}{2\left(m_1+m_2\right)}\nabla_\mathrm c^2\psi-\frac{\hbar^2}{2\mu}\nabla_r^2\psi+V\!\left(\mathbf r\right)\psi=E\psi.
\end{equation}
该方程成为可分离变量的方程.
将 $\psi$ 写作 $\psi_\mathrm c\!\left(\mathbf r_\mathrm c\right)\psi_r\!\left(\mathbf r\right)$,
得到
\begin{equation}
	-\frac{\hbar^2}{2\left(m_1+m_2\right)}\frac{\nabla_\mathrm c^2\psi_\mathrm c}{\psi_\mathrm c}-\frac{\hbar^2}{2\mu}\frac{\nabla_r^2\psi_r}{\psi_r}+V\!\left(\mathbf r\right)=E.
\end{equation}
于是有分离出的两个方程
\begin{equation}
	\label{eq:质心SE}
	-\frac{\hbar^2}{2\left(m_1+m_2\right)}\nabla_\mathrm c^2\psi_\mathrm c=E_\mathrm c\psi_\mathrm c,
\end{equation}
\begin{equation}
	\label{eq:相对SE}
	-\frac{\hbar^2}{2\mu}\nabla_r^2\psi_r+V\!\left(\mathbf r\right)\psi_r=E_r\psi_r.
\end{equation}
式 \ref{eq:质心SE} 描述质心运动, 式 \ref{eq:相对SE} 描述相对运动,
后者与处于势场 $V\!\left(\mathbf r\right)$ 中质量为 $\mu$ 的粒子的不含时 Schrödinger 方程相同.
式 \ref{eq:相对SE} 描述的系统的基态能量的相反数即为该二体系统的结合能.

设电子质量为 $m$, 则一个正电子和一个电子组成的系统的约化质量 $\mu=\frac m2$.
另一方面, 由于质子的质量远大于电子, 氢原子作为一个二体系统, 其约化质量 $\mu=m$.
因为正电子和电子之间的相互作用势能与电子与质子之间的相互作用势能是相同的,
所以式 \ref{eq:相对SE} 对电子偶素和氢原子两个模型的唯一区别就是 $\mu$.

记 $\psi^*\!\left(\mathbf r\right)$ 为氢原子的不含时 Schrödinger 方程的一个解,
则有
\begin{equation}
	\label{eq:氢原子特解}
	-\frac{\hbar^2}{2m}\nabla_r^2\psi^*\!\left(\mathbf r\right)-\frac{ke^2}{r}\psi^*\!\left(\mathbf r\right)=E^*\psi^*\!\left(\mathbf r\right).
\end{equation}
令 $\mathbf r'\coloneqq2\mathbf r$, 则有
\begin{equation}
	-\frac{\hbar^2}{2m}\cdot4\nabla_{r'}^2\psi^*\!\left(\frac{\mathbf r'}2\right)-2\frac{ke^2}{r'}\psi^*\!\left(\frac{\mathbf r'}2\right)=E^*\psi^*\!\left(\frac{\mathbf r'}2\right),
\end{equation}
或者
\begin{equation}
	\label{eq:变换为电子偶素的解}
	-\frac{\hbar^2}{2\frac m2}\nabla_{r'}^2\psi^*\!\left(\frac{\mathbf r'}2\right)-\frac{ke^2}{r'}\psi^*\!\left(\frac{\mathbf r'}2\right)=\frac{E^*}2\psi^*\!\left(\frac{\mathbf r'}2\right).
\end{equation}
式 \ref{eq:氢原子特解} 与式 \ref{eq:变换为电子偶素的解} 分别对应于式 \ref{eq:相对SE} 在 $\mu=m$ 与 $\mu=\frac m2$ 的情形.
因此, 对于氢原子的任意一个具有能量 $E^*$ 的定态解 $\psi^*\!\left(\mathbf r\right)$,
都有 $\psi^*\!\left(\frac{\mathbf r}2\right)$ 作为电子偶素的一个具有能量 $\frac{E^*}2$ 的定态解.
因此, 电子偶素的基态具有氢原子基态一半的能量.
即电子偶素的结合能为 $\SI{6.8}{eV}$.

\textit{另}: 此题可用 Bohr 模型解.

设电子与正电子在同一个半径为 $r$ 的圆轨道上运动, 速率恒定为 $v$.
则该系统的角动量为
\begin{equation}
	\label{eq:角动量}
	L=2mvr.
\end{equation}
对于基态, 有
\begin{equation}
	\label{eq:量子化条件}
	L=\hbar.
\end{equation}
另一方面, 静电力充当向心力, 得到
\begin{equation}
	\label{eq:向心力}
	\frac{mv^2}r=\frac{ke^2}{4r^2}.
\end{equation}
由式 \ref{eq:角动量}, 式 \ref{eq:量子化条件}, 式 \ref{eq:向心力} 解得
\begin{equation}
	v=\frac{ke^2}{2\hbar},\quad r=\frac{\hbar^2}{mke^2}.
\end{equation}
于是总能量
\begin{equation}
	E=2\cdot\frac12mv^2-\frac{ke^2}{2r}=-\frac{mk^2e^4}{4\hbar^2}=-\SI{6.8}{eV}.
\end{equation}
获得结合能 $\SI{6.8}{eV}$.

\subsection{电电圈}

容易得到半径为 $\rho'$ 的电流环轴线上的磁感应强度
\begin{equation}
	B_z=\frac{\mu_0I\rho'^2}{2\left(z^2+\rho'^2\right)^\frac32}.
\end{equation}
\begin{figure}[h!]
	\centering
	\begin{tikzpicture}
		\draw (5,0) circle (2);
		\node[anchor=north west] at (6.6,-1.2) {$r$};
		\draw[dashed] (0,0) -- (5,0) node[midway,anchor=north] {$R$};
		\draw[dashed] (5,0) node[anchor=north] {$\psi$} -- (3.8,0.9) node[midway,anchor=south] {$p$};
		\fill (3.8,0.9) circle (0.1);
	\end{tikzpicture}
	\caption{}
	\label{fig:电流环元}
\end{figure}
将整个圆环分解为多个电流环, 每个电流环的位置由圆环截面上的极坐标 $\left(p,\psi\right)$ 给出, 如图 \ref{fig:电流环元} 所示.
写出每个电流环元对轴线上磁场的贡献, 而后积分得
\begin{equation}
	B_z=\int_{p=0}^r\int_{\psi=0}^{2\pi}\frac{\mu_0\left(R-p\cos\psi\right)^2}{2\left(\left(z-p\sin\psi\right)^2+\left(R-p\cos\psi\right)^2\right)^\frac32}
	\frac{Q\omega\left(R-p\cos\psi\right)}{2\pi^2Rr^2}
	p\,\mathrm dp\,\mathrm d\psi.
\end{equation}
整理得
\begin{equation}
	B_z=\frac{\mu_0Q\omega}{4\pi^2Rr^2}
	\int_{p=0}^r\int_{\psi=0}^{2\pi}
	\frac{\left(R-p\cos\psi\right)^3p}{2\left(\left(z-p\sin\psi\right)^2+\left(R-p\cos\psi\right)^2\right)^\frac32}
	\,\mathrm dp\,\mathrm d\psi.
\end{equation}
由于积分号前有 $r^{-2}$ 因子, 而对 $p$ 的积分限是一阶量,
因此若要保证最终 $B_z$ 有 $r$ 的二阶量, 积分号内至少要展开至 $p$ 的三阶量.
展开后, 丢弃掉对 $\psi$ 积分后等于零的项, 得到
\begin{equation}
	B_z=\frac{\mu_0Q\omega}{4\pi^2Rr^2}
	\int_{p=0}^r\int_{\psi=0}^{2\pi}
	\left(\frac{R^3}{\left(R^2+z^2\right)^{\frac32}}p-
	\frac32\frac{Rz^2\left(3R^2-2z^2\right)\cos^2\psi+R^3\left(R^2-4z^2\right)\sin^2\psi}{\left(R^2+z^2\right)^\frac72}p^3\right)
	\mathrm dp\,\mathrm d\psi.
\end{equation}
将积分积出, 得到
\begin{equation}
	B_z=\frac{\mu_0Q\omega}{4\pi\left(R^2+z^2\right)^{\frac32}}
	\left(R^2-
	\frac{3}{8}\frac{R^2-2z^2}{R^2+z^2}r^2\right).
\end{equation}

在轴线上偏离轴线位置 $\rho$ 处的磁场的 $z$ 分量显然不包含 $\rho$ 的一阶项.
而 $B_\rho$ 对 $\rho$ 的一阶项可由静磁场 Gauss 定理易得
\begin{equation}
	B_\rho=-\frac12\frac{\mathrm dB_z}{\mathrm dz}\rho=
	\frac{3\mu_0Q\omega z\rho}{8\pi\left(R^2+z^2\right)^\frac52}\left(-R^2+\frac3{8}\frac{3R^2-2z^2}{R^2+z^2}r^2\right).
\end{equation}

\subsection{Theremin}

\mypara
电容的阻抗
\begin{equation}
	Z_C=\frac1{\mathrm i\omega C}=\frac1{2\pi\mathrm ifC}.
\end{equation}
通过率
\begin{equation}
	\left|\frac{Z_C}{R+Z_C}\right|=
	\frac1{\sqrt{1+4\pi^2C^2R^2f^2}}.
\end{equation}
通过率最高的频率的通过率为 $1$.
因此
\begin{equation}
	\eta\ge\frac1{\sqrt{1+4\pi^2C^2R^2f_{\mathrm{max}}^2}}.
\end{equation}

\mypara
考虑到 $U_0\ge U_{10},U_{20}$, 将二极管的伏安特性在小电压下展开到二阶, 得到
\begin{equation}
	I=I_0\left(\frac U{U_0}+\frac12\frac{U^2}{U_0^2}\right),
\end{equation}
其中 $U$ 为二极管上的分压.
二极管上的分压加上电阻上的分压等于 $U_1+U_2$, 于是
\begin{equation}
	U+RI_0\left(\frac{U}{U_0}+\frac{U^2}{2U_0^2}\right)=U_1+U_2.
\end{equation}
解得
\begin{equation}
	U=U_0\left(\sqrt{\left(\frac{U_0}{I_0R}+1\right)^2+\frac{U_1+U_2}{I_0R}}-\frac{U_0}{I_0R}-1\right).
\end{equation}
展开并保留到 $U_1+U_2$ 的二阶量, 得到
\begin{equation}
	U=U_0\left(\frac{U_1+U_2}{2\left(I_0R+U_0\right)}-\frac{I_0R\left(U_1+U_2\right)^2}{8\left(I_0R+U_0\right)^3}\right).
\end{equation}
上式是关于 $U_1,U_2$ 的二次函数,
将其展开后会有 $U_1,U_2,U_1^2,U_2^2,U_1U_2$ 各项.
前四种都是高频项 (因为 $f_1,f_2\gg f_{\mathrm{max}}$), 可以丢弃, 而
\begin{equation}
	U_1U_2=U_{10}U_{20}\cos\!\left(2\pi f_1t\right)\cos\!\left(2\pi f_2t\right)=
	\frac{U_{10}U_{20}}2\left(\cos\!\left(2\pi\left(f_1-f_2\right)t\right)+\cos\!\left(2\pi\left(f_1+f_2\right)t\right)\right),
\end{equation}
最终只有 $\cos\!\left(2\pi\left(f_1-f_2\right)t\right)$ 是低频项. 因此
\begin{equation}
	U\sim-\frac{U_0I_0RU_{10}U_{20}}{8\left(I_0R+U_0\right)^3}\cos\!\left(2\pi\left(f_1-f_2\right)t\right).
\end{equation}
电阻上的分压即为输出信号
\begin{equation}
	U_{\mathrm{out}}=U_1+U_2-U\sim
	\frac{U_0I_0RU_{10}U_{20}}{8\left(I_0R+U_0\right)^3}\cos\!\left(2\pi\left(f_1-f_2\right)t\right).
\end{equation}

\mypara
如图 \ref{fig:电路图} 所示.

\begin{figure}[h!]
	\centering
	\begin{circuitikz}
		\ctikzset{inductors/scale=1,quadpoles/transformer core/inner=1,quadpoles/transformer core/height=1.4, quadpoles/transformer core/width=0.6}
		\draw (2.4,0) node[transformer core,anchor=A1] (T1) {};
		\draw (T1.A2) to ++(-1.4,0) node[ground] {} to[capacitor] (1,0) to (T1.A1);
		\draw (2.4,-4) node[transformer core,anchor=A1] (T2) {};
		\draw (T2.A2) to ++(-1.4,0) node[ground] {} to[capacitor] (1,-4) to (T2.A1);
		\draw (T1.B1) to[diode] ++(2,0) node (N1) {} to[generic] ++(2,0) node (N2) {} to ++(1.5,0) node[ocirc] (end1) {};
		\draw (T2.B2) to ++(2,0) node (N3) {} to ++(2,0) node (N4) {} to ++(1.5,0) node[ocirc] (end2) {};
		\draw (T1.B2) to (T2.B1);
		\draw (N1) to[generic] (N3);
		\draw (N2) to[capacitor] (N4);
		\draw[<->] (end1) -- (end2) node[midway,fill=white] {音调信号};
		\node[anchor=north east] at (0.5,-1) {演奏者};
		\node[anchor=south east] at (0.5,-1) {音调天线};
		\node[anchor=south] at (1.7,-4) {参考音调振荡器};
		\node[anchor=south] at (1.7,0) {可变音调振荡器};
		\path (T2.B2) -- (N3) node[midway,anchor=north] {混频器};
		\path (N3) -- (N4) node[midway,anchor=north] {低通滤波器};
	\end{circuitikz}
	\caption{}
	\label{fig:电路图}
\end{figure}

\mypara
为计算两个圆柱形成的电容器单位长度上的电容, 运用电像法求解它们的电势,
电像为两根线电荷密度为 $\pm\lambda$ 的带电直线.
设圆柱 1 内的电像距离圆柱 1 的轴线为 $b_1$, 另一个为 $b_2$.

带电相反的两根均匀带电直线的等势面是到它们距离比值为定值的圆柱.
可以列出方程
\begin{equation}
	\begin{dcases}
		b_1\left(d-b_2\right)=R_1^2,\\
		b_2\left(d-b_1\right)=R_2^2.
	\end{dcases}
\end{equation}
由此解得 (一组解, 另一组解已舍)
\begin{equation}
	b_{1,2}=\frac d2\pm\frac{R_1^2-R_2^2}{2d}-\frac{\sqrt{\Delta}}{2d},
\end{equation}
其中 $\Delta\coloneqq\left(d-R_1-R_2\right)\left(d+R_1-R_2\right)\left(d-R_1+R_2\right)\left(d+R_1+R_2\right)$.

由此可以获得两个圆柱的电势差
\begin{equation}
	V_1-V_2=\frac\lambda{2\pi\varepsilon_0}\ln\frac{R_1-b_1}{d-R_1-b_2}
	-\frac\lambda{2\pi\varepsilon_0}\ln\frac{d-R_2-b_1}{R_2-b_2}.
\end{equation}
代入并整理得
\begin{equation}
	V_1-V_2=\frac{\lambda}{2\pi\varepsilon_0}\ln\frac{d^2-R_1^2-R_2^2-\sqrt{\Delta}}{2R_1R_2}.
\end{equation}
于是可以获得电容
\begin{equation}
	C=\frac{2\pi\varepsilon_0l}{\ln\frac{d^2-R_1^2-R_2^2-\sqrt{\Delta}}{2R_1R_2}}.
\end{equation}
$C$ 是随 $d$ 单调增的. 定义
$C_{\mathrm{min}}\coloneqq\left.C\right|_{d=d_{\mathrm{min}}}$ 以及 $C_{\mathrm{max}}$.

情形 1: 始终有 $f_1>f_2$.

此时有方程组
\begin{equation}
	\begin{dcases}
		\frac1{2\pi\sqrt{L_1C_{\mathrm{max}}}}-f_2=f_{\mathrm{min}},\\
		\frac1{2\pi\sqrt{L_1C_{\mathrm{min}}}}-f_2=f_{\mathrm{max}}.
	\end{dcases}
\end{equation}
解得
\begin{equation}
	L_1=\frac{\left(\sqrt{C_{\mathrm{max}}}-\sqrt{C_{\mathrm{min}}}\right)^2}{4\pi^2C_{\mathrm{max}}C_{\mathrm{min}}\left(f_{\mathrm{max}}-f_{\mathrm{min}}\right)^2},\qquad
	f_2=\frac{\sqrt{C_{\mathrm{min}}}f_{\mathrm{max}}-\sqrt{C_{\mathrm{max}}}f_{\mathrm{min}}}{\sqrt{C_{\mathrm{max}}}-\sqrt{C_{\mathrm{min}}}}.
\end{equation}

情形 2: 始终有 $f_1<f_2$.

\begin{equation}
	\begin{dcases}
		f_2-\frac1{2\pi\sqrt{L_1C_{\mathrm{min}}}}=f_{\mathrm{min}},\\
		f_2-\frac1{2\pi\sqrt{L_1C_{\mathrm{max}}}}=f_{\mathrm{max}}.
	\end{dcases}
\end{equation}

\begin{equation}
	L_1=\frac{\left(\sqrt{C_{\mathrm{max}}}-\sqrt{C_{\mathrm{min}}}\right)^2}{4\pi^2C_{\mathrm{max}}C_{\mathrm{min}}\left(f_{\mathrm{max}}-f_{\mathrm{min}}\right)^2},\qquad
	f_2=\frac{\sqrt{C_{\mathrm{max}}}f_{\mathrm{max}}-\sqrt{C_{\mathrm{min}}}f_{\mathrm{min}}}{\sqrt{C_{\mathrm{max}}}-\sqrt{C_{\mathrm{min}}}}.
\end{equation}

\end{document}
