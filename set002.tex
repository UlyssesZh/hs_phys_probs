\documentclass{article}
\usepackage[UTF8]{ctex}
\usepackage{tikz}
\usepackage{caption}
\usepackage{amsmath}
\usepackage{hyperref}
\usepackage{titlesec}
\usepackage{graphicx}
\usepackage[total={7in,10in}]{geometry}

\newcounter{para}
\newcommand\mypara{\par\refstepcounter{para}(\thepara)\space}
\titleformat{\section}[block]{\Large\bfseries\filcenter}{}{0em}{}
\renewcommand\thesection{}
\renewcommand\thesubsection{\setcounter{para}{0}\setcounter{equation}{0}第 \arabic{subsection} 题}

\usetikzlibrary{decorations.markings}

\title{hs\_phys\_probs 002}
\author{詹有丘}
\date{}

\begin{document}

\maketitle

\subsection{太空跳绳}
一根不可伸长的长度为 $l$, 质量为 $m$ 的均质软绳, 两端固定在间隔为 $b$ 的两点.
绳子以两个固定点的连线为轴以匀角速度 $\omega$ 转动.
忽略重力的影响而只考虑离心力.
转动过程中绳子的形状保持为一个平面图形不变.

\mypara
用平面坐标系中的方程描述绳子的形状.

\mypara
求绳子的角动量的大小.

\mypara
求固定点处对绳子的拉力的大小.

\subsection{地铁站闸机}
某地铁站的出入站闸机采用三锟闸设计.
三锟闸是这样一种装置:
考虑三维空间中的三根长度均为 $l$ 的细硬轻杆, 每根杆都有一端被固定在点 $O$ 处,
且它们两两之间的夹角被固定为 $\alpha$.
显然存在一条过 $O$ 的轴 $z$ 使得三锟闸绕 $z$ 轴有 $\frac{2\pi}3$ 旋转对称.
$z$ 轴与地面的夹角被适当地选取, 以至于三锟闸在初始状态可以与地面达成这样一种相对位形:
其中一根杆与地面平行, 另外两根杆的自由端的连线也与地面平行.
有一堵固定在地面上的墙, 其位置满足:
在初始状态下, 三锟闸的水平杆垂直于墙, 且墙面紧贴在水平杆的自由端.
将通过闸机的人简化为刚性长方体.
人通过闸机的过程中, 长方体推动三锟闸绕 $z$ 轴转动, 长方体的一个面紧贴地面, 另一个面紧贴墙面.
长方体足够高.

\mypara
求满足以下条件的长方体的最大宽度 $a_0$: 人能完全通过闸机, 且长方体的厚度可以任意大.

\mypara
接上问, 若长方体的宽度 $a>a_0$, 求满足以下条件的长方体的最大横截面积: 人能完全通过闸机.

\mypara
若长方体的宽度为 $a$, 人在完全通过闸机的过程中需要克服三种摩擦:
来自墙面和地面的滑动摩擦力 (大小恒定为 $f$),
来自杆的滑动摩擦力 (摩擦系数为 $\mu$),
来自三锟闸转轴的滑动摩擦力矩 (大小恒定为 $K$).
求人在缓慢地完全通过闸机的过程中, 来自杆的滑动摩擦耗散的能量为多少.

\subsection{Hohmann 转移轨道}
质量为 $m$ 的物体一开始绕着质量为 $M\gg m$ 的星体在半径为 $r_1$ 的圆轨道上运动.
某时其瞬间加速, 使速度方向不变, 速率增大 $\Delta v_1$, 进入椭圆轨道.
在远心点处, 其再次瞬间加速, 使速度方向不变, 速率增大 $\Delta v_2$,
进入半径为 $r_2=\alpha r_1$ 的圆轨道上运动.
证明使 $\Delta v_1+\Delta v_2$ 最大的 $\alpha$ 为
$5+4\sqrt7\cos\!\left(\frac13\arctan\frac{\sqrt3}{37}\right)$.

\subsection{电容势函数}
有一平行板电容器.
定义变量 $X$ 为极板间距, $Q$ 为一个极板上的电荷量大小,
$F$ 为极板间作用力, $V$ 为极板间的电势差.
电容 $C\!\left(X\right)$ 是已知函数 (不一定是反比例函数).
定义势函数 $U$ 为电容器储存的能量.

\mypara
证明 $\mathrm dU=V\mathrm dQ-F\mathrm dX$.

\mypara
证明 $\left(\frac{\partial V}{\partial F}\right)_Q=\left(\frac{\partial X}{\partial Q}\right)_F$.

\mypara
若 $C\!\left(X\right)$ 是反比例函数,
在 $F$-$X$ 图中分别作出等 $V$ 过程和等 $Q$ 过程的图像.

\newpage
\section{参考答案}

\subsection{太空跳绳}

\subsection{地铁站闸机}

\subsection{Holmann 转移轨道}

\subsection{电容势函数}

\end{document}
