\documentclass{article}
\usepackage[UTF8]{ctex}
\usepackage{tikz}
\usepackage{caption}
\usepackage{amsmath}
\usepackage{hyperref}
\usepackage{titlesec}
\usepackage{graphicx}
\usepackage[total={7in,10in}]{geometry}

\newcounter{para}
\newcommand\mypara{\par\refstepcounter{para}(\thepara)\space}
\titleformat{\section}[block]{\Large\bfseries\filcenter}{}{0em}{}
\renewcommand\thesection{}
\renewcommand\thesubsection{\setcounter{para}{0}\setcounter{equation}{0}第 \arabic{subsection} 题}

\usetikzlibrary{decorations.markings}

\title{hs\_phys\_probs 002}
\author{詹有丘}
\date{}

\begin{document}

\maketitle

\subsection{太空跳绳}

一根柔软的不可伸长的长度为 $l$, 质量为 $m$ 的均匀绳, 两端固定在间隔为 $b$ 的两点.
绳子以两个固定点的连线为轴以匀角速度 $\omega$ 转动.
忽略重力的影响而只考虑离心力.
转动过程中绳子保持其形状不变, 且任意时刻整根绳子在一个平面内.

\mypara
用平面坐标系中的方程描述绳子的形状.

\mypara
求绳子的角动量的大小.

\mypara
求固定点处对绳子的拉力的大小.

\newpage
\section{参考答案}

\subsection{太空跳绳}


\end{document}
